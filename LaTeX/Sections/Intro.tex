\section*{Introduction}
To effectively manage water resources, particularly in regions that lack a fresh, perennial surface water source, an understanding of long-term groundwater trends is essential \cite{jasechko2024rapid, gyawali2022quantifying, konikow2005groundwater, oki2006global}. Independent of human interaction, the water within aquifers behaves much like The Ouroboros does, undertaking in a state of cyclic renewal. However, global analysis of groundwater trends suggests that this natural balance has been disrupted, with Texas emerging as a hotspot for groundwater depletion \cite{scanlon2023global, oki2006global, vorosmarty2000global}. There are nine major Texas aquifers, which are the Ogallala, Edwards (Balcones Fault Zone), Edwards-Trinity Plateau, Carrizo-Wilcox, Gulf Coast, Pecos Valley, Seymour, Trinity and Hueco-Messila Bolson Aquifers \cite{twdb_major}. The change in groundwater levels in each of these aquifers though, vary significantly and several regions have been denoted as especially prone to declines in water level. These regions are the Texas Panhandle, Central Texas, and North-Central Texas \cite{chaudhuri2014long}. Each of these three regions corresponds with a major Texas aquifer and each have different water uses. The Ogallala Aquifer, which is one of the most well documented cases of groundwater decline globally, is located in the Texas Panhandle region, and has been depleted through the mining of fossil groundwater primarily for agricultural purposes \cite{scanlon2012overview, tewari2016management, konikow2005groundwater}.
