\begin{abstract}
Long-term trends in the nine major Texas aquifers, analyzed through the depths of wells drilled in a given year, show that groundwater wells have experienced increases in depth from 1920 to 2023 in all but one aquifer, with several aquifers characterized by significant depth increases of up to 2.56 ft/yr. These trends were interpreted from data downloaded from the Texas Water Development Board’s Groundwater Database (GWDB), which provides a large dataset of 63,472 usable well points defined by date of installation, latitude, longitude, and depth, allowing for robust analysis of well trends over both long and short periods (Texas Water Development Board, n.d.).

The depletion of groundwater creates significant risks for municipal water supply, agricultural production, and ecosystem stability statewide. By analyzing changes in water table depth using well depth as a proxy, this study identifies shifts in average well depths over time and highlights aquifers where these changes are most dramatic, indicating where sustainable management practices may be urgently needed. Aquifers determined to be notably unsustainable due to long-term deepening of well depths include the Ogallala, Edwards-Trinity Plateau, and Gulf Coast aquifers. The primary water uses for these aquifers have historically been irrigation, livestock, and municipal supply, respectively, making groundwater protection essential to preserve economic stability and public well-being (insert Chaudhuri ref)
\end{abstract}